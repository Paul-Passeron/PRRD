\section{Introduction}

L'article "When Separation Arithmetic is Enough" de  Filliâtre, Paskevich et Danvy présente une méthode pour la vérification formelle de programmes impératifs manipulant des structures de données récursives basées sur des pointeurs (ex: listes chaînées ou arbres binaires). Les auteurs proposent une solution qui réconcilie deux objectifs souvent contradictoires: maintenir des spécifications claires et compréhensibles tout en permettant une vérification largement automatisée par les solveurs SMT.\\


L'idée centrale de l'article consiste à projeter les structures récursives sur des séquences plates indexées par des entiers, transformant ainsi les propriétés de séparation (est-ce que plusieurs objets pointent vers le même endroit en mémoire) en contraintes d'arithmétique linéaires.


Pour commencer, nous implémentons les arbres binaires à l’aide d’une structure chaînée.

Chaque nœud est défini par une donnée (de type int dans notre cas), un pointeur vers son sous-arbre gauche et un pointeur vers son sous-arbre droit.

\begin{lstlisting}[language=C]
struct tree {
  struct tree *left;
  struct tree *right;
  elt_t dat;
};
\end{lstlisting}

L'étape suivante est d'implémenter l'algorithme de Morris.

L'algorithme de Morris sert à traverser les arbres binaires dans l'ordre sans utiliser de pile.

\bigskip

Voici la logique de cet algorithme, ci dessous :

\bigskip

On commence à la racine de l'arbre, puis on répète jusqu'au dernier nœud.

Pour chaque nœud courant :

\bigskip

S’il n’a pas de fils gauche :

\begin{itemize}
  \item $\rightarrow$ on le visite
  \item $\rightarrow$ on va à son fils droit
\end{itemize}

S’il a un fils gauche :

\begin{itemize}
  \item $\rightarrow$ on cherche son prédécesseur : le nœud le plus à droite du sous-arbre gauche
\end{itemize}

Si ce prédécesseur ne pointe pas encore vers le nœud courant :

\begin{itemize}
  \item $\rightarrow$ on crée un lien temporaire vers le nœud courant
  \item $\rightarrow$ on va à gauche
\end{itemize}

Si ce prédécesseur pointe déjà vers le nœud courant :

\begin{itemize}
  \item $\rightarrow$ on supprime le lien
  \item $\rightarrow$ on visite le nœud courant
  \item $\rightarrow$ on va à droite
\end{itemize}

\bigskip

Pour implémenter cet algorithme, on crée quatre fonctions.

\bigskip

La fonction \texttt{void visit(tree\_t t);}  
sert à visiter chaque nœud, en affichant sa valeur.

\bigskip

La fonction \texttt{void traversal(tree\_t t);}  
sert de point d’entrée et lance le parcours de Morris à partir de la racine de l’arbre.

\bigskip

La fonction \texttt{bool warp(tree\_t t, tree\_t q);}  
sert à créer le lien entre le nœud le plus à droite du sous-arbre gauche et le nœud courant s'il n'existe pas encore (et retourne \texttt{true}), sinon elle l'enlève et retourne \texttt{false}.

\bigskip

La fonction \texttt{void morris\_visit(tree\_t t);}  
est la fonction principale de l'algorithme.  
Elle applique directement la logique de Morris décrite au-dessus en appelant les fonctions \texttt{warp} et \texttt{visit}.

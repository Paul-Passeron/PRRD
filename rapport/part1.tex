\section{Introduction}

L'article "When Separation Arithmetic is Enough" de  Filliâtre, Paskevich et Danvy présente une méthode pour la vérification formelle de programmes impératifs manipulant des structures de données récursives basées sur des pointeurs (ex: listes chaînées ou arbres binaires). Les auteurs proposent une solution qui réconcilie deux objectifs souvent contradictoires: maintenir des spécificationsclaires et compréhensibles tout en permettant une vérification largement automatisée par les solveurs SMT.

L'idée centrale de l'article consiste à projeter les structures récursives sur des séquences plates indexées par des entiers, transformant ainsi les propriétés de séparation (est-ce que plusieurs objets pointent vers le même endroit en mémoire) en contraintes d'arithmétique linéaires.

Dans ce rapport, nous nous concentrerons sur les quatres premières sections de l'article, qui présentent la méthode à travers trois exemples de plus en plus complexes: l'inversion classique de liste, une variante inversant les valeurs en espace constant, et l'algorithme de Morris pour le parcours d'arbres binaires. Nous avons choisi de ne pas traiter les sections 5 et 6, la première présentant un exemple bonus issu de la compétition VerifyThis 2021 qui reprend essentiellement les concepts déjà illustrés, et la seconde constituant une revue de littérature de travaux connexes qui sort du cadre de cet article.
